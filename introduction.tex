% $Id:introduction.tex  $
% !TEX root = main.tex
\chapter{Introduction}

\section{Context}

\textit{Biometrics} introduce to the recognition of human by their characteristics and are used to identify individuals in groups. Traditional methods of identification involve passwords, however new techniques of identification are preferred over them for security reasons. Biometric systems are divided on the basis of medium used for authentication. and technologies as Face Recognition, Iris Recognition, Palm Recognition And Voice Recognition has been used in each medium depending on multiple factors \cite{kaur2014review}.

The \textbf{Voice recognition systems} often face various types of \textit{spoofing attacks}, with speech synthesis and speech conversion being the most common ones. Speech synthesis refers to the generation of spoken language by machines based on written input \cite{oxforddictionary}. On the other hand, speech conversion involves modifying speech waveforms to convert non-linguistic information while preserving the linguistic information \cite{stylianou1998continuous}. Such fraudulent activities pose a significant threat to the system security, as they may deceive the biometric speech recognition system into accepting these attacks as authentic \cite{xue2023physiological}. Spoofing attacks can lead to unauthorized access, financial fraud, or identity theft, and how these can have serious implications for individuals and organizations alike. In the last decade with the appearance of new deep learning \textbf{\textit{generative architectures}} such as \ac{GAN} \cite{goodfellow2020generative} and \ac{VAE} \cite{kingma2013auto} the potential to been affected by spoofing attacks has increased, and the need for improve systems that helps to recognize these attacks is urgent, although in the last decade the advance in the anti-spoofing models has been significant. The research field currently lacks exploration in the impact of prosodic features on the overall performance of \ac{AS} models. Specifically, there is limited research on how accent features, such as vowel space area and speaking rate, can influence the effectiveness of these models. This knowledge gap highlights the need for further investigation and understanding in this area.

\section{Problem Statement}
Previous studies using English pre-trained models have shown that anti-spoof models trained on English datasets exhibit a significant drop in performance when applied to Spanish datasets \cite{tamayo2022voice}. It is important to note that even with state-of-the-art architectures, multi-language capabilities cannot be assumed even within the same language. And that's why we propose to explore how the accent features affect the overall performance and test multiple architectures to analyze their impact in the performance of each model.

\section{Justification}
Current anti-spoofing mechanisms used to classify audio signals as spoof or bonafide often rely on systems that do not take into account multiple accents or languages either in their training datasets or in their feature extraction modules. This limits the effectiveness of these mechanisms, as they may not be able to accurately detect spoofing attacks in scenarios where the attacker or the genuine user has different accents than the system was trained. Therefore, there is a need for a more robust anti-spoofing architecture that can detect spoofing attacks regardless of the accent used by the attacker or the user. Moreover, since the significant lack of research dedicated to understanding the factors behind the degradation of system performance due to speech accents \cite{tamayo2022voice} we can not assume that there is no way to make the models more robust in terms of detect an accent base spoofing attack and analyze what is the impact in an \acl{AS} attack. Incorporating this aspects into consideration could result in the development of a more robust model capable of effectively classifying spoof attacks across a wide range of accents or even languages.

\section{Objectives}

\subsection{General objective}

\textbf{General objective}: To compare and evaluate the effect of various combinations of back-ends, front-ends, accents, and languages in the development of antispoof systems.

%Comparar y evaluar el efecto de diversas combinaciones de back-ends, front-ends, acentos y lenguajes en el desarrollo de sistemas de antispoofing. 


%To develop an anti-spoofing system that can accurately detect spoofing attacks regardless of the attacker's accent or language. 

%Based on the back end and front-end architecture, with the front-end responsible for processing the audio signals and extracting features, and the back-end responsible for classifying the audio signals as either spoof or genuine (bona-fide). Therefore, the objectives of this project are:

\subsection{Specific Objectives}

\begin{enumerate}
    \item Implement multiple front-end models, including \acs{CQCC}, \acs{MFCC}, \acs{LFCC}, Spectrogram, and Self-supervised models, that can extract relevant features from the audio signals, and analyze the relationship between accents and feature extraction.
    \item Implement multiple back-end models using \acs{LCNN}, \acs{Bi-LSTM}, \acs{GAP} Layers that can classify the audio signals as either spoof or bona-fide and analyze the impact of the different back-end layers on the overall performance of the system.
    \item Compares the performance of different front-end and back-end architectures combinations on multilanguage  and multiaccent training sets. 
    \item Analyzes the impact of accent and language diversity on accuracy and provides insights regarding the most suitable architecture for each specific language and accent.
\end{enumerate}




\endinput

